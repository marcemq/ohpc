\documentclass[letterpaper]{article}
\usepackage{common/ohpc-doc}
\setcounter{secnumdepth}{5}
\setcounter{tocdepth}{5}

% Include git variables
\input{vc.tex}

% Define Base OS and other local macros
\newcommand{\baseOS}{CentOS7.3}
\newcommand{\OSRepo}{CentOS\_7.3}
\newcommand{\OSTree}{CentOS\_7}
\newcommand{\OSTag}{el7}
\newcommand{\baseos}{centos7.3}
\newcommand{\baseosshort}{centos7}
\newcommand{\provisioner}{xCAT}
\newcommand{\rms}{SLURM}
\newcommand{\rmsshort}{slurm}
\newcommand{\arch}{x86\_64}

% Define package manager commands
\newcommand{\pkgmgr}{yum}
\newcommand{\addrepo}{wget -P /etc/yum.repos.d}
\newcommand{\chrootaddrepo}{wget -P \$CHROOT/etc/yum.repos.d}
\newcommand{\clean}{yum clean expire-cache}
\newcommand{\chrootclean}{yum --installroot=\$CHROOT clean expire-cache}
\newcommand{\install}{yum -y install}
\newcommand{\chrootinstall}{yum -y --installroot=\$CHROOT install}
\newcommand{\groupinstall}{yum -y groupinstall}
\newcommand{\groupchrootinstall}{yum -y --installroot=\$CHROOT groupinstall}
\newcommand{\remove}{yum -y remove}
\newcommand{\upgrade}{yum -y upgrade}
\newcommand{\chrootupgrade}{yum -y --installroot=\$CHROOT upgrade}
\newcommand{\tftppkg}{syslinux-tftpboot}
\newcommand{\beegfsrepo}{https://www.beegfs.io/release/beegfs\_6/dists/beegfs-rhel7.repo}

% boolean for os-specific formatting
\toggletrue{isCentOS}
\toggletrue{isCentOS_ww_slurm_x86}
\toggletrue{isx86}
\toggletrue{isxCAT}

\begin{document}
\graphicspath{{common/figures/}}
\thispagestyle{empty}

% Title Page
\input{common/title}
% Disclaimer 
\input{common/legal} 

\newpage
\tableofcontents
\newpage

% Introduction  --------------------------------------------------

\section{Introduction} \label{sec:introduction}
\input{common/install_header}
\input{common/intro} \\

\input{common/base_edition/edition}
\input{common/audience}
\input{common/requirements}
\input{common/inputs}

% begin_ohpc_run
% ohpc_validation_newline
% ohpc_validation_comment Verify OpenHPC repository has been enabled before proceeding
% ohpc_validation_newline
% ohpc_command yum repolist | grep -q OpenHPC
% ohpc_command if [ $? -ne 0 ];then
% ohpc_command    echo "Error: OpenHPC repository must be enabled locally"
% ohpc_command    exit 1
% ohpc_command fi
% end_ohpc_run

% Base Operating System --------------------------------------------

\section{Install Base Operating System (BOS)}
\input{common/bos}

\clearpage 
% begin_ohpc_run
% ohpc_validation_newline
% ohpc_validation_comment Disable firewall 
\begin{lstlisting}[language=bash,keywords={}]
[sms](*\#*) systemctl disable firewalld
[sms](*\#*) systemctl stop firewalld
\end{lstlisting}
% end_ohpc_run

% ------------------------------------------------------------------

\section{Install \OHPC{} Components} \label{sec:basic_install}
\input{common/install_ohpc_components_intro.tex}

\subsection{Enable \OHPC{} repository for local use} \label{sec:enable_repo}
\input{common/enable_ohpc_repo}
\subsection{Enable \xCAT{} repository for local use} \label{sec:enable_xcat}
\input{common/enable_xcat_repo}

\noindent \xCAT{} has a number of dependencies that are required for
installation that are housed in separate public repositories for various
distributions. To enable for local use, issue the following:

% begin_ohpc_run
\begin{lstlisting}[language=bash,keywords={},basicstyle=\fontencoding{T1}\fontsize{8.0}{10}\ttfamily,literate={ARCH}{\arch{}}1 {-}{-}1]
[sms](*\#*)  (*\addrepo*) https://xcat.org/files/xcat/repos/yum/xcat-dep/rh7/ARCH/xCAT-dep.repo
\end{lstlisting}
% end_ohpc_run

In addition to the \OHPC{} and \xCAT{} package repositories, the {\em master} host also
requires access to the standard base OS distro repositories in order to resolve
necessary dependencies. For \baseOS{}, the requirements are to have access to
both the base OS and EPEL repositories for which mirrors are freely available online:

\begin{itemize*}
\item CentOS-7 - Base 7.3.1611
  (e.g. \href{http://mirror.centos.org/centos-7/7/os/x86\_64}
             {\color{blue}{http://mirror.centos.org/centos-7/7/os/x86\_64}} )
\item EPEL 7 (e.g. \href{http://download.fedoraproject.org/pub/epel/7/x86\_64}
                        {\color{blue}{http://download.fedoraproject.org/pub/epel/7/x86\_64}} )
\end{itemize*}

\noindent The public EPEL repository will be enabled automatically upon installation of the 
\texttt{ohpc-release} package. Note that this requires the CentOS Extras
repository, which is shipped with CentOS and is enabled by default.

\input{common/automation}


\subsection{Add provisioning services on {\em master} node} \label{sec:add_provisioning}
\input{common/install_provisioning_xcat_intro}
\input{common/enable_pxe}
\input{common/time}

\subsection{Add resource management services on {\em master} node} \label{sec:add_rm}
\input{common/install_slurm}

\subsection{Add \InfiniBand{} support services on {\em master} node} \label{sec:add_ofed}
\input{common/ibsupport_sms_centos}

\vspace*{-0.15cm}
\subsection{Complete basic \xCAT{} setup for {\em master} node} \label{sec:setup_xcat}
\input{common/xcat_setup}


\subsection{Define {\em compute} image for provisioning}
\input{common/xcat_mkchroot_centos}

\subsubsection{Add \OHPC{} components} \label{sec:add_components}
\input{common/add_to_compute_chroot_xcat_intro}

%\newpage
% begin_ohpc_run
% ohpc_validation_comment Add OpenHPC components to compute instance
\begin{lstlisting}[language=bash,literate={-}{-}1,keywords={},upquote=true]
# Add Slurm client support meta-package
[sms](*\#*) (*\chrootinstall*) ohpc-slurm-client

# Add IB support and enable
[sms](*\#*) (*\groupchrootinstall*) "InfiniBand Support"
[sms](*\#*) (*\chrootinstall*) infinipath-psm
[sms](*\#*) chroot $CHROOT systemctl enable rdma

# Add Network Time Protocol (NTP) support
[sms](*\#*) (*\chrootinstall*) ntp

# Add kernel drivers
[sms](*\#*) (*\chrootinstall*) kernel

# Include modules user environment
[sms](*\#*) (*\chrootinstall*) lmod-ohpc
\end{lstlisting}
% end_ohpc_run


\subsubsection{Customize system configuration} \label{sec:master_customization}
\input{common/xcat_chroot_customize_centos}

% Additional commands when additional computes are requested

% begin_ohpc_run
% ohpc_validation_newline
% ohpc_validation_comment Update basic slurm configuration if additional computes defined
% ohpc_command if [ ${num_computes} -gt 4 ];then
% ohpc_command    perl -pi -e "s/^NodeName=(\S+)/NodeName=${compute_prefix}[1-${num_computes}]/" /etc/slurm/slurm.conf
% ohpc_command    perl -pi -e "s/^PartitionName=normal Nodes=(\S+)/PartitionName=normal Nodes=${compute_prefix}[1-${num_computes}]/" /etc/slurm/slurm.conf

% ohpc_command    perl -pi -e "s/^NodeName=(\S+)/NodeName=${compute_prefix}[1-${num_computes}]/" $CHROOT/etc/slurm/slurm.conf
% ohpc_command    perl -pi -e "s/^PartitionName=normal Nodes=(\S+)/PartitionName=normal Nodes=${compute_prefix}[1-${num_computes}]/" $CHROOT/etc/slurm/slurm.conf
% ohpc_command fi
% end_ohpc_run

%\clearpage
\subsubsection{Additional Customization ({\em optional})} \label{sec:addl_customizations}
This section highlights common additional customizations that can {\em
optionally} be applied to the local cluster environment. These customizations
include:

\begin{multicols}{2}
\begin{itemize*}
\item Increase memlock limits

\nottoggle{ispbs}{\item Restrict ssh access to compute resources}

\item Add \beegfs{} client
\item Add \Lustre{} client

\iftoggle{isWarewulf}{\item Enable syslog forwarding}

\item Add \Nagios{} Core monitoring
\item Add \Ganglia{} monitoring
\item Add \Sensys{} monitoring
\item Add \clustershell{}
\item Add \mrsh{}
\item Add \genders{}
%%\item Add \powerman{}
\item Add \conman{}  
\end{itemize*}
\end{multicols}

\noindent Details on the steps required for each of these customizations are
discussed further in the following sections.


\paragraph{Increase locked memory limits}
\input{common/memlimits}

\paragraph{Enable ssh control via resource manager} 
\input{common/slurm_pam}

\paragraph{Add \Lustre{} client} \label{sec:lustre_client}
\input{common/lustre-client}
\input{common/lustre-client-centos}
\input{common/lustre-client-post}

\paragraph{Add \Nagios{} monitoring}
\input{common/nagios}

\vspace*{0.4cm}
\paragraph{Add \Ganglia{} monitoring}
\input{common/ganglia}

\paragraph{Add \Sensys{} monitoring}
\Sensys{} provides resilient and scalable monitoring for resource utilization
and state of node health, collecting all the data in a database for subsequent
analysis. Sensys includes several loadable plugins that monitor various metrics
related to different features present in each node like temperature, voltage,
power usage, memory, disk and process information.\\

\noindent Sensys has two daemons called {\em orcmd} and {\em orcmsched} that run
at root level. Daemon {\em orcmd} can be defined to run as an {\em aggregator}
or a {\em compute node} monitoring agent.
An {\em aggregator} is an interface that receives all the telemetry collected by
the {\em compute nodes} monitoring agents and stores it to a database, {\em
orcmsched} is an agent that tracks the presence of all the {\em orcmd} daemons
and their connections in the cluster.\\
The mapping of nodes to the roles provided by \Sensys{} must be defined by the
user via the configuration file {\em orcm-site.xml} located under the {\em etc}
directory at \Sensys{} installation path.
\Sensys{} requires a postgresql database enabled in the cluster, it can be
placed in the same node where the {\em aggregator} daemon is going to be
executed. \\

\noindent The simplest way to install, setup and enable \Sensys{} in a cluster
where the {\em SMS} features a postgresql database server and a {\em nfs}
filesystem is accessible by all {\em compute nodes} in the cluster is shown
below:

\begin{lstlisting}[language=bash,keywords={},upquote=true]
# Install Sensys meta-package
[sms](*\#*) (*\install*) ohpc-sensys
# Add the default installation path /opt/sensys to the /etc/exports file
[sms](*\#*) echo "/opt/sensys/ *(ro,fsid=13)" >> /etc/exports
[sms](*\#*) SMS_IP=<replace-with-ip-of-sms>; echo $SMS_IP":/opt/sensys/
/opt/sensys/ nfs nfsvers=3 0 0" >> $CHROOT/etc/fstab

# Apply changes to nfs service
[sms](*\#*) exportfs -a
[sms](*\#*) systemctl restart nfs

# Setup a database for the collected data
[sms](*\#*) (*\install*) postgresql
[sms](*\#*) su -u postgres createuser -P <db_user>
[sms](*\#*) su -u postgres createdb --owner <db_user> <db_name>
[sms](*\#*) psql -U <db_user> -W -f /opt/sensys/share/db-schema/sensys.sql <db_name>

# Update the default configuration file, by adding your scheduler and aggregator hostnames
# and use a regex to update the compute nodes
[sms](*\#*) sed -i 's/SMS/<replace-with-hostname-of-sms>/' /opt/sensys/etc/orcm-site.xml
[sms](*\#*) sed -i 's/agg01/<replace-with-hostname-of-aggregator>/' /opt/sensys/etc/orcm-site.xml
[sms](*\#*) sed -i 's/cn01/<replace-with-compute-nodes-prefix>[2:00-49]/' /opt/sensys/etc/orcm-site.xml

# Example: monitoring coretemp with a time period of 30 seconds
# Launch Sensys scheduler in the background
[sms](*\#*) /opt/sensys/bin/orcmsched &
# Launch Sensys on aggregator
[sms](*\#*) /opt/sensys/bin/orcmd -omca sensor heartbeat,coretemp \
        -omca sensor_base_sample_rate 30 \
        -omca db_postgres_uri localhost:5432 \
        -omca db_postgres_user <db_user>:<db_password> \
        -omca db_postgres_database <db_name> &
# Launch Sensys on compute nodes
[sms](*\#*) /opt/sensys/bin/orcmd -omca sensor heartbeat,coretemp \
        -omca sensor_base_sample_rate 30 &
\end{lstlisting}

\noindent  Once enabled and running, \Sensys{} should start filling up the
database with the monitored data, which can be queried at any time.

\begin{lstlisting}[language=bash,keywords={},upquote=true]
# Log into database
[sms](*\#*) psql -d <db_name> -U <db_user>
# Query data_sample_raw table
(*\$*) SELECT * FROM data_sample_raw;
\end{lstlisting}

\noindent \Sensys{} has many more sensors and options available and those can be
reviewed \href{https://intel-ctrlsys.github.io/sensys}{\color{blue}{here}}.


\paragraph{Add \clustershell{}}
\input{common/clustershell}

\paragraph{Add \mrsh{}}
\input{common/mrsh}

\paragraph{Add \genders{}}
\input{common/genders}

\paragraph{Add \conman{}} \label{sec:add_conman}
\input{common/conman}

\subsubsection{Identify files for synchronization} \label{sec:file_import}
\input{common/import_xcat_files}
\input{common/import_xcat_files_slurm}
\input{common/finalize_xcat_provisioning}
\input{common/add_xcat_hosts_intro}

%%%\subsubsection{Optional kernel arguments} \label{sec:optional_kargs}
%%%\input{common/conman_post}

%\vspace*{-0.25cm}
\subsection{Boot compute nodes} \label{sec:boot_computes}
\input{common/reset_computes} 

%\vspace*{-0.50cm}
\section{Install \OHPC{} Development Components}
\input{common/dev_intro.tex}

\vspace*{-0.15cm}
\subsection{Development Tools} \label{sec:install_dev_tools}
\input{common/dev_tools}

\vspace*{-0.15cm}
\subsection{Compilers} \label{sec:install_compilers}
\input{common/compilers}

%\clearpage
\subsection{MPI Stacks} \label{sec:mpi}
\input{common/mpi}

\subsection{Performance Tools} \label{sec:install_perf_tools}
\input{common/perf_tools}

\subsection{Setup default development environment}
\input{common/default_dev}

%\vspace*{0.2cm}
\subsection{3rd Party Libraries and Tools} \label{sec:3rdparty}
\input{common/third_party_libs_intro}

\input{common/third_party_libs}
\vspace*{0.1cm}
\input{common/third_party_mpi_libs_x86}

\subsection{Optional Development Tool Builds} \label{sec:3rdparty_intel}
\input{common/pxse_enabled_builds}

\section{Resource Manager Startup} \label{sec:rms_startup}
\input{common/slurm_startup}

\section{Run a Test Job} \label{sec:test_job}
\input{common/xcat_slurm_test_job}

\clearpage
\appendix
%\section*{Appendices}
{\bf \LARGE \centerline{Appendices}} \vspace*{0.2cm}

\addcontentsline{toc}{section}{Appendices}
\renewcommand{\thesubsection}{\Alph{subsection}}

\input{common/automation_appendix}
\input{common/upgrade}
\input{common/test_suite}
\input{common/customization_appendix_x86}
\input{manifest}
\input{common/signature}


\end{document}

